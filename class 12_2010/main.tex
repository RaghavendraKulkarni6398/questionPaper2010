\documentclass[journal,12pt,twocolumn]{IEEEtran}
%
\usepackage{setspace}
\usepackage{gensymb}
%\doublespacing
\singlespacing

%\usepackage{graphicx}
%\usepackage{amssymb}
%\usepackage{relsize}
\usepackage[cmex10]{amsmath}
%\usepackage{amsthm}
%\interdisplaylinepenalty=2500
%\savesymbol{iint}
%\usepackage{txfonts}
%\restoresymbol{TXF}{iint}
%\usepackage{wasysym}
\usepackage{amsthm}
%\usepackage{iithtlc}
\usepackage{mathrsfs}
\usepackage{txfonts}
\usepackage{stfloats}
\usepackage{bm}
\usepackage{cite}
\usepackage{cases}
\usepackage{subfig}
%\usepackage{xtab}
\usepackage{longtable}
\usepackage{multirow}
%\usepackage{algorithm}
%\usepackage{algpseudocode}
\usepackage{enumitem}
\usepackage{mathtools}
\usepackage{tikz}
\usepackage{circuitikz}
\usepackage{verbatim}
%\usepackage{tfrupee}
\usepackage[breaklinks=true]{hyperref}
%\usepackage{stmaryrd}
\usepackage{tkz-euclide} % loads  TikZ and tkz-base
%\usetkzobj{all}
\usepackage{listings}
    \usepackage{color}                                            %%
    \usepackage{array}                                            %%
    \usepackage{longtable}                                        %%
    \usepackage{calc}                                             %%
    \usepackage{multirow}                                         %%
    \usepackage{hhline}                                           %%
    \usepackage{ifthen}                                           %%
  %optionally (for landscape tables embedded in another document): %%
    \usepackage{lscape}     
\usepackage{multicol}
\usepackage{chngcntr}
%\usepackage{enumerate}

%\usepackage{wasysym}
%\newcounter{MYtempeqncnt}
\DeclareMathOperator*{\Res}{Res}
%\renewcommand{\baselinestretch}{2}
\renewcommand\thesection{\arabic{section}}
\renewcommand\thesubsection{\thesection.\arabic{subsection}}
\renewcommand\thesubsubsection{\thesubsection.\arabic{subsubsection}}

\renewcommand\thesectiondis{\arabic{section}}
\renewcommand\thesubsectiondis{\thesectiondis.\arabic{subsection}}
\renewcommand\thesubsubsectiondis{\thesubsectiondis.\arabic{subsubsection}}

% correct bad hyphenation here
\hyphenation{op-tical net-works semi-conduc-tor}
\def\inputGnumericTable{}                                 %%

\lstset{
%language=C,
frame=single, 
breaklines=true,
columns=fullflexible
}
%\lstset{
%language=tex,
%frame=single, 
%breaklines=true
%}

\begin{document}

\newtheorem{theorem}{Theorem}[section]
\newtheorem{problem}{Problem}
\newtheorem{proposition}{Proposition}[section]
\newtheorem{lemma}{Lemma}[section]
\newtheorem{corollary}[theorem]{Corollary}
\newtheorem{example}{Example}[section]
\newtheorem{definition}[problem]{Definition}
%\newtheorem{thm}{Theorem}[section] 
%\newtheorem{defn}[thm]{Definition}
%\newtheorem{algorithm}{Algorithm}[section]
%\newtheorem{cor}{Corollary}
\newcommand{\BEQA}{\begin{eqnarray}}
\newcommand{\EEQA}{\end{eqnarray}}
\newcommand{\define}{\stackrel{\triangle}{=}}
\bibliographystyle{IEEEtran}
%\bibliographystyle{ieeetr}
\providecommand{\mbf}{\mathbf}
\providecommand{\pr}[1]{\ensuremath{\Pr\left(#1\right)}}
\providecommand{\qfunc}[1]{\ensuremath{Q\left(#1\right)}}
\providecommand{\sbrak}[1]{\ensuremath{{}\left[#1\right]}}
\providecommand{\lsbrak}[1]{\ensuremath{{}\left[#1\right.}}
\providecommand{\rsbrak}[1]{\ensuremath{{}\left.#1\right]}}
\providecommand{\brak}[1]{\ensuremath{\left(#1\right)}}
\providecommand{\lbrak}[1]{\ensuremath{\left(#1\right.}}
\providecommand{\rbrak}[1]{\ensuremath{\left.#1\right)}}
\providecommand{\cbrak}[1]{\ensuremath{\left\{#1\right\}}}
\providecommand{\lcbrak}[1]{\ensuremath{\left\{#1\right.}}
\providecommand{\rcbrak}[1]{\ensuremath{\left.#1\right\}}}
\theoremstyle{remark}
\newtheorem{rem}{Remark}
\newcommand{\sgn}{\mathop{\mathrm{sgn}}}
\providecommand{\abs}[1]{\left\vert#1\right\vert}
\providecommand{\res}[1]{\Res\displaylimits_{#1}} 
\providecommand{\norm}[1]{\left\lVert#1\right\rVert}
%\providecommand{\norm}[1]{\lVert#1\rVert}
\providecommand{\mtx}[1]{\mathbf{#1}}
\providecommand{\mean}[1]{E\left[ #1 \right]}
\providecommand{\fourier}{\overset{\mathcal{F}}{ \rightleftharpoons}}
%\providecommand{\hilbert}{\overset{\mathcal{H}}{ \rightleftharpoons}}
\providecommand{\system}{\overset{\mathcal{H}}{ \longleftrightarrow}}
	%\newcommand{\solution}[2]{\textbf{Solution:}{#1}}
\newcommand{\solution}{\noindent \textbf{Solution: }}
\newcommand{\cosec}{\,\text{cosec}\,}
\providecommand{\dec}[2]{\ensuremath{\overset{#1}{\underset{#2}{\gtrless}}}}
\newcommand{\myvec}[1]{\ensuremath{\begin{pmatrix}#1\end{pmatrix}}}
\newcommand{\mydet}[1]{\ensuremath{\begin{vmatrix}#1\end{vmatrix}}}
%\numberwithin{equation}{section}
%\numberwithin{equation}{subsection}
%\numberwithin{problem}{section}
%\numberwithin{definition}{section}
\makeatletter
\@addtoreset{figure}{problem}
\makeatother
\let\StandardTheFigure\thefigure
\let\vec\mathbf
%\renewcommand{\thefigure}{\theproblem.\arabic{figure}}
\renewcommand{\thefigure}{\theproblem}
%\setlist[enumerate,1]{before=\renewcommand\theequation{\theenumi.\arabic{equation}}
%\counterwithin{equation}{enumi}
%\renewcommand{\theequation}{\arabic{subsection}.\arabic{equation}}
\def\putbox#1#2#3{\makebox[0in][l]{\makebox[#1][l]{}\raisebox{\baselineskip}[0in][0in]{\raisebox{#2}[0in][0in]{#3}}}}
     \def\rightbox#1{\makebox[0in][r]{#1}}
     \def\centbox#1{\makebox[0in]{#1}}
     \def\topbox#1{\raisebox{-\baselineskip}[0in][0in]{#1}}
     \def\midbox#1{\raisebox{-0.5\baselineskip}[0in][0in]{#1}}
\vspace{3cm}
\title{
%	\logo{
MATHEMATICS
%	}
}
\author{ G V V Sharma$^{*}$% <-this % stops a space
	\thanks{*The author is with the Department
		of Electrical Engineering, Indian Institute of Technology, Hyderabad
		502285 India e-mail:  gadepall@iith.ac.in. All content in this manual is released under GNU GPL.  Free and open source.}
	
}	
%\title{
%	\logo{Matrix Analysis through Octave}{\begin{center}\includegraphics[scale=.24]{tlc}\end{center}}{}{HAMDSP}
%}
% paper title
% can use linebreaks \\ within to get better formatting as desired
%\title{Matrix Analysis through Octave}
%
%
% author names and IEEE memberships
% note positions of commas and nonbreaking spaces ( ~ ) LaTeX will not break
% a structure at a ~ so this keeps an author's name from being broken across
% two lines.
% use \thanks{} to gain access to the first footnote area
% a separate \thanks must be used for each paragraph as LaTeX2e's \thanks
% was not built to handle multiple paragraphs
%
%\author{<-this % stops a space
%\thanks{}}
%}
% note the % following the last \IEEEmembership and also \thanks - 
% these prevent an unwanted space from occurring between the last author name
% and the end of the author line. i.e., if you had this:
% 
% \author{....lastname \thanks{...} \thanks{...} }
%                     ^------------^------------^----Do not want these spaces!
%
% a space would be appended to the last name and could cause every name on that
% line to be shifted left slightly. This is one of those "LaTeX things". For
% instance, "\textbf{A} \textbf{B}" will typeset as "A B" not "AB". To get
% "AB" then you have to do: "\textbf{A}\textbf{B}"
% \thanks is no different in this regard, so shield the last } of each \thanks
% that ends a line with a % and do not let a space in before the next \thanks.
% Spaces after \IEEEmembership other than the last one are OK (and needed) as
% you are supposed to have spaces between the names. For what it is worth,
% this is a minor point as most people would not even notice if the said evil
% space somehow managed to creep in.
% The paper headers
%\markboth{Journal of \LaTeX\ Class Files,~Vol.~6, No.~1, January~2007}%
%{Shell \MakeLowercase{\textit{et al.}}: Bare Demo of IEEEtran.cls for Journals}
% The only time the second header will appear is for the odd numbered pages
% after the title page when using the twoside option.
% 
% *** Note that you probably will NOT want to include the author's ***
% *** name in the headers of peer review papers.                   ***
% You can use \ifCLASSOPTIONpeerreview for conditional compilation here if
% you desire.
% If you want to put a publisher's ID mark on the page you can do it like
% this:
%\IEEEpubid{0000--0000/00\$00.00~\copyright~2007 IEEE}
% Remember, if you use this you must call \IEEEpubidadjcol in the second
% column for its text to clear the IEEEpubid mark.
% make the title area


\maketitle
\newpage
%\tableofcontents
\bigskip
\renewcommand{\thefigure}{\theenumi}
\renewcommand{\thetable}{\theenumi}
%\renewcommand{\theequation}{\theenumi}
%\begin{abstract}
%%\boldmath
%In this letter, an algorithm for evaluating the exact analytical bit error rate  (BER)  for the piecewise linear (PL) combiner for  multiple relays is presented. Previous results were available only for upto three relays. The algorithm is unique in the sense that  the actual mathematical expressions, that are prohibitively large, need not be explicitly obtained. The diversity gain due to multiple relays is shown through plots of the analytical BER, well supported by simulations. 
%
%\end{abstract}
% IEEEtran.cls defaults to using nonbold math in the Abstract.
% This preserves the distinction between vectors and scalars. However,
% if the journal you are submitting to favors bold math in the abstract,
% then you can use LaTeX's standard command \boldmath at the very start
% of the abstract to achieve this. Many IEEE journals frown on math
% in the abstract anyway.
% Note that keywords are not normally used for peerreview papers.
%\begin{IEEEkeywords}
%Cooperative diversity, decode and forward, piecewise linear
%\end{IEEEkeywords}
% For peer review papers, you can put extra information on the cover
% page as needed:
% \ifCLASSOPTIONpeerreview
% \begin{center} \bfseries EDICS Category: 3-BBND \end{center}
% \fi
%
% For peerreview papers, this IEEEtran command inserts a page break and
% creates the second title. It will be ignored for other modes.
%\IEEEpeerreviewmaketitle
%\begin{abstract}
%This manual includes \LaTeX figures.
%book provides an introduction to optimization  based on the NCERT textbooks from Class 6-12.  Links to sample Python codes are available in the text.  
%\end{abstract}
Download 
\begin{lstlisting}
https://github.com/RaghavendraKulkarni6398/QuestionPaper2010
\end{lstlisting}

\begin{document}
\vspace{0.15in}
\begin{center}
\parbox{1in}{
{\textsc{\textbf{Section A} }}}}
\end{center}
\vspace{0.15in}
\hrule 
\vspace{0.1in}
\begin{enumerate}

\item 1.If f: $R\longrightarrow R$ be defined by $\left f(x) = (3x^3)^\frac{1}{3}\right$, then find $f(x)$  \\
\vspace{0.2in}
\item Write the principal value of $sec^-1(-2)$.
\vspace{0.2in}
\item What positive value of $x$ makes the following pair of determinants equal ? 
\[
\begin{vmatrix}
2x & 3 \\ 
5 & x
\end{vmatrix}
\], \[
\begin{vmatrix}
16 & 3 \\ 
5 & 2
\end{vmatrix}
\]
\vspace{0.2in}
\item Evaluate :$\int sec^2 (7-4x)dx$
\vspace{0.2in}
\item Write the adjoint of the following matrix:
\begin{bmatrix}
2 & -1\\
4 & 3
\end{bmatrix}
\vspace{0.2in}
\item Write the value of the following integral:
\[ \int_{\frac{\pi}{2}}^{\frac{\pi}{2}} sin^5 x \,dx \]
\vspace{0.2in}
\item A is a square matrix of order 3 and $|A|= 7$. Write the value of $|adj|$
\vspace{0.2in}
\item  Write the distance of the following plane from the origin: $2xy + 2z + 1 = 0$
\vspace{0.2in} 
\item  Write a vector of magnitude 9 units in the direction of vector $-2i+j+2k$.
\vspace{0.3in}
\item Find $\lambda$ if $(2i+ 6j + 14k)\times (i - \lambdaĴ + 7k) = o$.
 \vspace{0.3in} 
\item  A family has 2 children. Find the probability that both are boys, if it is known that

(i) the elder child is a boy.

(ii) at least one of the children is a boy
\vspace{0.3in}      
\item  Show that the relation S in the set $A = [x \epsilon Z:0\leq x \leq 12)$ given by $S= [(a, b):a, b \epsilon Z$, $|a - b|$ is divisible by 4 is an equivalence relation. Find the set of all elements related to 1
\vspace{0.3in}    
\item Prove the following:
$\Declarenewcommand{\taninv}{\tan^{-1}}x + \Declarenewcommand{\taninv}{\tan^{-1}}x (\frac{2x}{1-x^2})$ = $\Declarenewcommand{\taninv}{\tan^{-1}} (\frac{3x-x^3}{1-3x^2})$ (OR) Prove the following:
$cos[\Declarenewcommand{\taninv}{\tan^{-1}}\{\sin(\Declarenewcommand{\taninv}{\cot^{-1}}x)\}]=\sqrt{\frac{1+x^2}{2+x^2}}$
\vspace{0.3in} 
\item Express the following matrix as the sum of a symmetric and skew symmetric matrix, and verify your result:
\newline
\begin{bmatrix}
3 & -2 & -4\\
3 & -2 & -5\\
-1 & 1 & 2
\end{bmatrix}
\vspace{0.3in}
\item  If $\overrightarrow{a} = i+j+k$,$\overrightarrow{b} = 4i-2j+3k$
 and $\overrightarrow{c} = i-2j+k$ find a vector of magnitude 6 units which is parallel to the vector $2a-b+3c$.\newline(OR) Let $\overrightarrow{a} = i+4j+2k$,$\overrightarrow{b} = 3i-2j+7k$
 and $\overrightarrow{c} = 2i-j+4k$ find a vector$\overrightarrow{d}$ which is perpendicular to both \overrightarrow{a} and $\overrightarrow{b} and $\overrightarrow{c} . $\overrightarrow{d} = 18$.
 \pagebreak  
\vspace{0.3in}
\item  Find the points on the line $\frac{x+2}{3}=\frac{y+1}{2}=\frac{z-3}{2}$at a distance of 5 units from the point $P(1, 3, 3)$.\newline(OR) Find the distance of the point $P(6, 5, 9)$ from the plane determined by the points $A(3, -1, 2)$, $B(5, 2, 4)$ and $C(-1, -1, 6)$. 
\vspace{0.3in} 
\item Solve the following differential equation:
\newline $(x^2-1)\frac{dy}{dx} + 2xy = \frac{1}{x^2-1} |x|$
\vspace{0.1in} 
\newline OR
\vspace{0.1in} 
\newline $\sqrt{1+x^2+y^2+x^2y^2}+xy\frac{dy}{dx}=0$
\vspace{0.3in} 
\item  Show that the differential equation $(x-y)\frac{dy}{dx} = x + 2y$, is homogeneous and solve it equation
\vspace{0.3in} 
\item  Evaluate the following:
\newline\[ \int_\frac{x+2}{\sqrt{(x-2)(x-3)}} \,dx \]
\vspace{0.3in}
\item  Evaluate the following:
\[ \int_{1}^{2} \frac{5x^2}{x^2+4x+3} \,dx \]
\vspace{0.3in} 
\item If $y=e^{a\Declarenewcommand{\sininv}{\sin^{-1}}x}$, $-1\leq x \leq 1$ then show that $(1-x^2)\frac{d^2y}{dx^2} - x\frac{dy}{dx}-a^2y = 0$
\vspace{0.3in} 
\item If $y=\Declarenewcommand{\cosinv}{\cos^{-1}}\left(\frac{3x+4\sqrt{1-x^2}}{5}\right)$, find $\frac{dy}{dx}$ 
\vspace{0.3in} 
\item Using properties of determinants, prove the following:
\newline\begin{vmatrix}
x & x^2 & 1+px^3\\
y & y^2 & 1+py^3\\
z & z^2 & 1+pz^3
\end{vmatrix}
= $(1+pxyz)(x-y)(y-z)(z-x)$
\newline (OR)
\newline Find the inverse of the following matrix using elementary operations :
\newline\begin{pmatrix}
1 & 2 & -2\\
-1 & 3 & 0\\
0 & -2 & 1
\end{pmatrix}
\vspace{0.3in}
\item A bag contains 4 balls. Two balls are drawn at random, and are found to be white. What is the probability that all balls are white?
\vspace{0.3in} 
\item One kind of cake requires 300 g of flour and 15 g of fat, another kind of cake requires 150g of flour and 30g of fat. Find the maximum number of cakes which can be made from 7.5 kg of flour and 600g of fat, assuming that there is no shortage of the other in gradients used in making the cakes. Make it as an L.P.P. and solve it graphically. 

\vspace{0.3in}

\item Find the coordinates of the foot of the perpendicular and the perpendicular distance of the point P(3, 2, 1) from the plane $2xy + z + 1 = 0$. Find also, the image of the point in the plane

\vspace{0.2in}

\item Find the area of the circle $4x^2+4y^2=9$ which the interior to the parabola.$x^2 = 4y$
\newline (OR)
\newline Using integration, find the area of the triangle ABC, coordinates of whose vertices are A(4, 1), B(6, 6) and C(8, 4).

\vspace{0.15in}

\item If the length of three sides of a trapezium other than the base is 10 cm each, find the area of the trapezium, when it is maximum.
\vspace{0.2in}

\item Find the intervals in which the following function is

(a) strictly increasing,

(b) strictly decreasing.
\vspace{0.2in}

\end{questions}
\end{document}